\documentclass[12pt,a4,english,finnish,pdflatex%,handout
]{beamer}
\definecolor{MyGreen}{RGB}{50, 120, 50}
\usecolortheme[named=MyGreen]{structure}
\usefonttheme{serif}

\usepackage{babel}
\usepackage[utf8]{inputenc}
\usepackage[T1]{fontenc}
\usepackage{amsmath,amssymb} 
\usepackage{animate}
\usepackage{multimedia}

\usepackage{natbib}
\bibpunct[: ]{(}{)}{,}{}{}{;}

\usepackage{tikz}

\usepackage{tipa}

\usepackage{hyperref}

\usepackage{graphicx}

\setbeamertemplate{navigation symbols}{}

\graphicspath{{figures/}}

\setlength{\leftmargini}{0pt}
\setlength{\leftmarginii}{1em}

\newcommand{\kommentti}[1]{
  {\bf[#1]}
}

\author{Pertti Palo} 

\begin{document}
\title{Analysing Articulatory Data with Vector Norms and Related Methods}
\date{9 Oct 2023} 

\frame{\titlepage
  \centering
} 

\frame{\frametitle{Outline}
  \begin{itemize}
    \item I will then talk about what we can and cannot do with these methods in
    time domain analysis of articulatory data these days. I will take short side
    trips to look at similar methods applied to other articulatory data. We have
    looked at tongue splines and lip videos and I will discuss what kind of
    challenges and understanding has resulted from those attempts. Finally,
    analysing 3D/4D ultrasound has been a recent major focus, but unfortunately
    frame rate issues are not so easy to solve.
    \item I will finish the talk by discussing why MRI would be very interesting
    to analyse with these methods -- or adapted versions of them -- and which
    definite and possible challenges one might come across.
  \end{itemize}
}


\frame{\frametitle{Introduction: The why}
  \begin{itemize}
  \item Pre-speech articulation is interesting from several points of view, but
  analysing ultrasound videos manually is not great.
  \item In my thesis I concentrated on timing of utterance onset in both
  acoustics and articulation \citep{Palo-MeasuringPrespeechArticulation-2019}.
  \item The data was high-speed tongue ultrasound from a delayed naming
  experiment -- specifically one using the Rastle instructions
  \citep{RastleEtAl-CharacterizingMotorExecution-2005}.
  \item 2D ultrasound has good time resolution: 80-120 fps in today's examples.
  \end{itemize} 

 	\centering
	\includegraphics[width=\textwidth]{figures/stages_of_naming.jpg}
}


\frame{\frametitle{Introduction: The why}
  \begin{itemize}
  \item When trying to identify movement onset in greyscale videos with a lot of
  speckle 'noise', it doesn't take long to grow a desire for an easier way.
  \item The speckle 'noise' maybe caused by a number of factors including bubbles in the 
  acoustic gel between the chin and the probe, and more interestingly changes in internal 
  structures of tissues -- such as muscle fibres tensing and relaxing.
  \end{itemize} 
  
  
}

\frame{\frametitle{What is being imaged by tongue ultrasound?}
	\centering
	\includegraphics[width=0.45\linewidth]{figures/DA_and_P1_overlay_2}
	\includegraphics[width=0.45\linewidth]{figures/DA_and_P1_overlay_3}
}

\frame{\frametitle{Where is the tongue?}

	\centering
	\hspace*{-.75cm}
	\includegraphics[width=1.125\linewidth]{P2_lemon_1st_frame_spline.pdf}
}



\frame{\frametitle{Pixel Difference (PD)}
	\begin{itemize}
	\item The first tool out of the box happened to work adequately --  and so for my
	thesis I used Euclidean distance or $l2$-norm to identify articulatory onsets:
	\end{itemize}
	\begin{equation*}
		l2(t+0.5) = \sqrt{\sum_{i, j} (x(i,j,t+1) - x(i,j,t))²}
	\end{equation*}


	\centering
	\includegraphics[height=.5\textheight]{figures/pd_caught.jpg}
}

\frame{\frametitle{Pixel Difference (PD)}
	\begin{itemize}
		\item PD is usually calculated on uninterpolated (probe-return) ultrasound 
		data (a) as opposed to interpolated (human-readable) data (b).
	\end{itemize}
	\centering
	\includegraphics[height=.7\textheight]{figures/raw_and_interpolated.jpg}

}

\frame{\frametitle{Pixel Difference (PD)}
	\centering
	\includegraphics[height=.8\textheight]{figures/pixel_difference_demo_noise_square_sum_tall.png}
}

\frame{\frametitle{Introduction: The how}
	\begin{itemize}
	\item de-interlaced video and the zigzag: time steps
	\item splines and the sparseness of the data
	\item PD on ultrasound and time resolution
	\item Choosing a metric
	\end{itemize}
}


\frame{\frametitle{PD on de-interlaced videos}
	% A slowed down and up-close example of interlacing by Wikipedia user Grayshi.
	Lip video with camera attached to the ultrasound helmet. De-interlaced at 59.94 fps.
	\centering
	\includegraphics[width=0.7\linewidth]{figures/File127_pd_ult_and_vid_ts1_2.png}
	
}

\frame{\frametitle{PD on de-interlaced videos}
	% A slowed down and up-close example of interlacing by Wikipedia user Grayshi.
	Taking a different time step:	
	\centering
	\includegraphics[width=\linewidth]{figures/d1_and_d3.pdf}

}

\frame{\frametitle{Tongue splines and problems from sparseness}

	\textbf{Raw ultrasound:} 
	\begin{itemize}
		\item Typically 63x406 pixels per frame.
		\item Individual pixel's fluctuations get averaged out.
	\end{itemize}

	\vspace{.25cm}
	\textbf{Tongue splines:}
	\begin{itemize}
		\item Typically control points per frame.
		\item Individual point's fluctuations may end up driving the data.
	\end{itemize}
	
	\vspace{.25cm}
	\centering
	\hspace*{-.75cm}
	\includegraphics[width=1.125\linewidth]{P2_lemon_1st_frame_spline.pdf}
	
	
}

\frame{\frametitle{Tongue splines and problems from sparseness}

	\begin{itemize}
		\item Longer time step and averaging produce better results, but still with limits.
		\item Here and in the next slide ANND and MPBPD have been calculated with time step 3 and smoothed with a moving average filter with a 5
		frame window.
	\end{itemize}

	\centering
	\includegraphics[width=.8\linewidth]{figures/File120_annd_pd}
}

\frame{\frametitle{Tongue splines and problems from sparseness}

	\begin{itemize}
	\item Choice of metric can help, but not with everything.
	\end{itemize}

	\centering
	\includegraphics[width=\linewidth]{figures/File120_annd_mpbpd_pd}
	
}

\frame{\frametitle{3D/4D ultrasound}
	
	\begin{itemize}
		\item Capturing a 3D frame takes a lot longer and we only have access to interpolated data.
		\item The images are always interpolated.
		\item In analysis even on good (lucky) samples onset and gesture recognition becomes difficult.
	\end{itemize}
	
	\begin{figure}
	\centering
	\includegraphics[width=.9\linewidth]{figures/PD_on_3D_cropped.png}	
	\end{figure}
	
}

\frame{\frametitle{PD on Raw vs Interpolated 2D data}
	
	\centering
	\includegraphics[width=.8\linewidth]{figures/raw_interpolated_2D}
}

\frame{\frametitle{PD on data with artificially lowered frame rate}
	\centering
	\includegraphics[width=.8\linewidth]{figures/raw_interpolated_2D_step5}
	
}

\frame{\frametitle{In the works: Choosing the metric for PD}
	\begin{itemize}
		\item PD has so far usually been calculated as the Euclidean distance or $l2$-norm.
		\item We've recently been looking at principled ways of selecting the norm for a given data source -- such as 2D ultrasound -- from the different $lp$-norms where $p \in \mathopen]0,\inf\mathclose[$. 
		\item It looks like the optimal norm for 2D ultrasound is $l1$ (or close to it):
	\end{itemize}	
	\begin{equation*}
		l1(t+0.5) = \sum_{i, j} |x(i,j,t+1) - x(i,j,t)|
	\end{equation*}
}

\frame{\frametitle{So how about MRI then?}
	
	\begin{itemize}
		\item Frame rate can be a problem.
		\item If there are systematic changes frame-to-frame caused by the imaging and reconstruction these may show up in PD analysis.
		\item Testing different norms, and applying larger time steps and smoothing if needed will hopefully lead to analysable data.
	\end{itemize}
	
}

\frame{\frametitle{References}
  
\bibliographystyle{apalike}
\bibliography{science_combined.bib}

}

\frame{
  \centering
  {
    \bf \Large 
    \usebeamercolor[fg]{title}
    Extra material
    
    \vfill
%    \includegraphics[height=1.5cm]{figures/aalto_logo} 
  }
}


\frame{\frametitle{Delayed naming results: Acoustics}
	\begin{center}
		\vspace*{-.5cm}
		\hspace*{-1cm}
		\includegraphics[width=\textwidth]{figures/OD_vs_AcRT.jpg}
	\end{center}
	Medianised within participant, over several repetitions and over
	the vowels \textipa{/a,i,O/}. Over all analysable n = 1386: 439 from P1, 672 from P3, and
	275 from P4.  }

\frame{\frametitle{Delayed naming results: Articulatory to Acoustic Interval}
	\begin{center}
		\vspace*{-.5cm}
		\hspace*{-1cm}
		\includegraphics[width=\textwidth]{figures/OD_vs_AAI.jpg}
	\end{center}
	Medianised within participant, over several repetitions and over
	the vowels \textipa{/a,i,O/}. Over all analysable n = 1386: 439 from P1, 672 from P3,
	and 275 from P4.
}

\frame{\frametitle{Theory: Effect of OD on AAI}
	\begin{itemize}
		\item As the Onset Duration (OD) gets longer, Articulatory to Acoustic Interval (AAI) shortens.
		\item First three lines represent individual utterances, final line is a conceptual model of the effect of continuously lengthening OD.6
	\end{itemize}
	\begin{center}
		\hspace*{-.5cm}
		\includegraphics[width=1.1\textwidth]{effect_of_OD.drawio.png}
	\end{center}
}

\frame{\frametitle{Theory: Effect of articulatory rate on AAI}
	\begin{itemize}
		\item If we keep the utterance content constant but vary articulation rate, all parts (AAI, OD, and acoustic word) get longer as articulation rate goes down.
	\end{itemize}
	\begin{center}
		\hspace*{-.5cm}
		\includegraphics[width=\textwidth]{effect_of_RhymeDur.drawio.png}
	\end{center}
}

\end{document}


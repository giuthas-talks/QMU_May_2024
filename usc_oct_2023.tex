\documentclass[12pt,a4,english,finnish,pdflatex%,handout
]{beamer}
\definecolor{MyGreen}{RGB}{50, 120, 50}
\usecolortheme[named=MyGreen]{structure}

\usepackage{babel}
\usepackage[utf8]{inputenc}
\usepackage[T1]{fontenc}
\usepackage{amsmath,amssymb} 
\usepackage{animate}
\usepackage{multimedia}

\usepackage{natbib}
\bibpunct[: ]{(}{)}{,}{}{}{;}

\usepackage{tikz}

\usepackage{tipa}

\usepackage{hyperref}

\setbeamertemplate{navigation symbols}{}

\graphicspath{{figures/}}

\setlength{\leftmargini}{0pt}
\setlength{\leftmarginii}{1em}

\newcommand{\kommentti}[1]{
  {\bf[#1]}
}



\begin{document}
\title{Analysing Image Sequences and Other Articulatory Data\\
Using Vector Norms and Related Methods 
  \\~\\
  \kommentti{This is a template}
}
\author{Pertti Palo} 
\date{9 Oct 2023} 

\frame{\titlepage
  \centering
} 

\frame{\frametitle{Outline}
  \begin{itemize}
    \item This talk starts with a short intro on how I got started with Pixel
    Difference or using vector norms to analyse overall change in ultrasound
    (and other) video sequences. It all began with needing to find a quick way
    of identifying articulatory onsets in ultrasound recordings of a delayed
    naming experiment (specifically one using the Rastle instructions -- see
    Rastle et al. 2005). The first version of Pixel Difference was simply
    Euclidean distance.
    \item I will then talk about what we can and cannot do with these methods in
    time domain analysis of articulatory data these days. I will take short side
    trips to look at similar methods applied to other articulatory data. We have
    looked at tongue splines and lip videos and I will discuss what kind of
    challenges and understanding has resulted from those attempts. Finally,
    analysing 3D/4D ultrasound has been a recent major focus, but unfortunately
    frame rate issues are not so easy to solve.
    \item I will finish the talk by discussing why MRI would be very interesting
    to analyse with these methods -- or adapted versions of them -- and which
    definite and possible challenges one might come across. .
  \item stuff
  \end{itemize}
}


\frame{\frametitle{Introduction: The why}
  \begin{itemize}
  \item Pre-speech articulation is interesting from several points of view, but
  analysing ultrasound videos manually is not great \citep{Palo-MeasuringPrespeechArticulation-2019}.
  \item stuff
  \end{itemize}
}


\frame{\frametitle{References}
  
\bibliographystyle{apalike}
\bibliography{science_combined.bib}

}

\frame{
  \centering
  {
    \bf \Large 
    \usebeamercolor[fg]{title}
    Something else i.e. section title
    
    \vfill
%    \includegraphics[height=1.5cm]{figures/aalto_logo} 
  }
}


\end{document}

